\documentclass[10.5pt,a4paper]{article}

% Margin lebih sempit agar space lebih banyak
\usepackage[left=1cm, right=1cm, top=1cm, bottom=1cm]{geometry}
\usepackage{titlesec}
\usepackage{enumitem}
\usepackage{hyperref}
\usepackage{tabularx}
\usepackage{xcolor}
\usepackage{colortbl}
\usepackage{array} % untuk tabular yang rapi
\usepackage{pifont} % di preamble
\usepackage{hyperref}
%---------------------------------------------------------------------------------

\usepackage{titlesec} % Allows creating custom \sections

% Format of the section titles
\titleformat{\section}
{\color[HTML]{1F4E79}\bfseries\scshape\large\raggedright} % Biru profesional
{}{0em}{}
\titlespacing{\section}{0pt}{12pt}{5pt}
%----------------------------------------------------

% Gunakan fontspec untuk font modern seperti Raleway
\usepackage[default]{raleway}

% Hilangkan spasi antar paragraf
\setlength{\parskip}{0pt}
\setlength{\parindent}{0pt}

\begin{document}

\begin{center}
  {\Huge{Rendy Ramadhana Wahyudi}}\\[0.5em]
    
  {\scshape \large Passionate Geophysicist \quad | \quad Seismic Data Enthusiast}\\[0.5em]

  
  \rule{\textwidth}{0.4pt}\\[0.5em]
  \normalsize
  Bandung, West Java \quad | \quad 
  \href{mailto:rendy.rw.rw@gmail.com}{rendy.rw.rw@gmail.com} \quad | \quad
  \href{https://rendy-rw.github.io/}{rendy-rw.github.io} \quad | \quad
  \href{https://www.linkedin.com/in/rendywahyudi}{linkedin.com/in/rendywahyudi}
\end{center}

\section*{Professional Summary}
Fast-learning and flexible young professional with a passion for continuous learning and a solution-oriented mindset. Eager to leverage education and training in subsurface geoscience, combining a hands-on approach with strong attention to detail to support the effective transformation of seismic data into actionable insights. Demonstrates a proven ability to work collaboratively in teams or independently, with excellent verbal and written communication skills.
\section*{Areas of Expertise}
\noindent
\renewcommand{\arraystretch}{1.2} % lebih padat
\begin{tabularx}{\textwidth}{>{\raggedright\arraybackslash}X >{\raggedright\arraybackslash}X >{\raggedright\arraybackslash}X}
  \rowcolor[HTML]{EDF4FB} 
  \ding{51} Seismic Processing  & \ding{51} GIS and Mapping Tools & \ding{51} Project Planning  \\
  \rowcolor[HTML]{FFFFFF} 
  \ding{51} Geophysical Data Interpretation & \ding{51} Data Management and QC & \ding{51} Signal Processing \\
  \rowcolor[HTML]{EDF4FB} 
  \ding{51} Programming for Geoscience & \ding{51} Geostatistics & \ding{51} Scientific Writing \\
\end{tabularx}


\section*{Education}


\setlength{\fboxsep}{0pt}%
\colorbox[HTML]{EDF4FB}{%
  \parbox{\linewidth}{%
    \textbf{Master of Science in Physics Engineering}, Universitas Padjadjaran, West Java \quad | \quad Nov 2023 - Aug 2025 (Expected)%
  }%
}
\begin{itemize}[left=1.5em, noitemsep, topsep=0pt]
    \item \textbf{Capstone Project}: \textit{Developing seismic-based disaster detection systems aligned with regional priorities.}
\end{itemize}

\vspace{0.5em}

\setlength{\fboxsep}{0pt}%
\colorbox[HTML]{EDF4FB}{%
  \parbox{\linewidth}{%
    \textbf{Bachelor of Science in Geophysical Engineering}, Universitas Padjadjaran, West Java \quad | \quad Aug 2019 - Jul 2023%
  }%
}
\begin{itemize}[left=1.5em, noitemsep, topsep=0pt]
  \item \textbf{Capstone Project}: \textit{Prototype development of IoT-based distributed seismic sensor network}, published at \linebreak DOI: \href{https://doi.org/10.2991/978-2-38476-283-5_7}{10.2991/978-2-38476-283-5\_7}.
\end{itemize}





\section*{Leadership Experience}
\noindent 
\setlength{\fboxsep}{0pt}%
\colorbox[HTML]{EDF4FB}{%
  \parbox{\linewidth}{%
    \textbf{Program Coordinator, Village Empowerment}, HiMA GEOFISIKA PEDRA, UNPAD \quad | \quad Sep 2021 - Feb 2022%
  }%
}

Coordinated long-term rural outreach, trained disaster awareness to 100+ residents.
\begin{itemize}[left=1.5em, noitemsep, topsep=0pt]
  \item Conducted social mapping across 50+ households and collaborated to co-design impactful educational agendas.
  \item Delivered earthquake preparedness workshops attended by 100+ residents, enhancing local resilience.
\end{itemize}

\vspace{0.5em}

\noindent 
\setlength{\fboxsep}{0pt}%
\colorbox[HTML]{EDF4FB}{%
  \parbox{\linewidth}{%
    \textbf{Chair, Independence Day Celebration Committee}, Community Resident Group \quad | \quad Aug 2021%
  }%
}

Organized community-wide event in less than a week, fostering strong local engagement.
\begin{itemize}[left=1.5em, noitemsep, topsep=0pt]
  \item Led a diverse volunteer team of 15 to organize an inclusive Independence Day celebration.
  \item Fostered participation from 50+ households, creating a sense of belonging across the community.
\end{itemize}



\section*{Professional Experience}

\setlength{\fboxsep}{0pt}%
\colorbox[HTML]{EDF4FB}{%
  \parbox{\linewidth}{%
    \textbf{Geophysical Engineer, Freelance},  PT. Surya Brinka Persada, West Java \quad | \quad Sep 2024 - Present%
  }%
}
\begin{itemize}[left=1.5em, noitemsep, topsep=0pt]
    \item Led a team of 5 in survey design, data acquisition, processing, and preliminary interpretation, while managing and monitoring equipment to ensure accurate and reliable geophysical datasets.
    \item Earned positive feedback for optimizing project execution, resulting in up to 50\% savings in both time and budget.  
\end{itemize}

\vspace{0.5em}

\setlength{\fboxsep}{0pt}%
\colorbox[HTML]{EDF4FB}{%
  \parbox{\linewidth}{%
    \textbf{Geoscience Surveyor, Volunteer}, Yayasan Sedekah Air, Jakarta \quad | \quad Dec 2022 - Present%
  }%
}

\begin{itemize}[left=1.5em, noitemsep, topsep=0pt]
    \item Liaised with local communities as key stakeholders, communicating technical assessments of potential groundwater sources while managing expectations to ensure alignment with both community needs and environmental standards.
    \item Conducted geological and geophysical surveys using the resistivity method, including data quality control, real-time monitoring, and interpretation to ensure reliable subsurface characterization.
    \item Provided viable water sources for 50 to 100 households and public facilities (e.g., places of worship and schools) by utilizing accurate, reliable geophysical data to guide effective groundwater development.
  \end{itemize}

\section*{Certificates \& Grant}

\textbf{TOEFL ITP Score: 527}, Universitas Indonesia \quad | \quad Jul 2024 \\
\textbf{Copyright: Educational Documentary Film}, Kemenkumham \quad | \quad Apr 2023\\
\textbf{Graduate Research Grant}, Kemendikbud \quad | \quad Aug 2024 \\
\textbf{Graduate Scholarship Awardee}, JFLS, Disdik Jabar \quad | \quad Feb 2025

\section*{Technical Skills}

\textbf{Programming}: Python, C++, MATLAB  \quad | \quad
\textbf{Geoscience Software}: Res2Dinv, Petrel, Geopsy, QGIS, Surfer  \quad | \quad
\textbf{Documentation \& Reporting}: LaTeX, Scientific Writing \& Technical Reporting  \quad | \quad
\textbf{Project Management \& Collaboration}: Trello, Notion  \quad | \quad
\textbf{Operating System}: Windows, Linux (Ubuntu)

\end{document}