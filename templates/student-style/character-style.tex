\documentclass[11pt,a4paper]{article}

% Margin lebih sempit agar space lebih banyak
\usepackage[left=1.5cm, right=1.5cm, top=1.5cm, bottom=1.5cm]{geometry}
\usepackage{titlesec}
\usepackage{enumitem}
\usepackage{hyperref}
\usepackage{xcolor}
\usepackage{colortbl}
\usepackage{array} % untuk tabular yang rapi
\usepackage{pifont} % di preamble
\usepackage{hyperref}
%---------------------------------------------------------------------------------

\usepackage{titlesec} % Allows creating custom \sections

% Format of the section titles
\titleformat{\section}
  {\color[HTML]{666666}\bfseries\scshape\large\raggedright} % Warna abu gelap + tebal + small caps
  {}{0em}{}
\titlespacing{\section}{0pt}{12pt}{5pt}
%----------------------------------------------------

% Gunakan fontspec untuk font modern seperti Raleway
\usepackage[default]{raleway}

% Hilangkan spasi antar paragraf
\setlength{\parskip}{0pt}
\setlength{\parindent}{0pt}

\begin{document}

\begin{center}
    {\LARGE \textbf{Rendy Ramadhana Wahyudi}}\\
    \vspace{0.2em}
    Subsurface analyst | Tech-forward geophysics enthusiast
    \rule{\textwidth}{0.4pt}

    \small
    Bandung, West Java \quad | \quad 
    \href{mailto:rendy.rw.rw@gmail.com}{rendy.rw.rw@gmail.com} \quad | \quad
    \href{https://rendy-rw.github.io/}{rendy-rw.github.io} \quad | \quad
    \href{https://www.linkedin.com/in/rendywahyudi}{linkedin.com/in/rendywahyudi}
\end{center}


\section*{Professional Summary}
Driven geophysics enthusiast with strong curiosity for the subsurface and a commitment to energy innovation. 
Skilled in seismic interpretation and data analysis to support hydrocarbon exploration. 
Proven ability to solve complex geoscience problems through creativity and collaboration. 
Strong verbal and written communication, with an analytical mindset and hands-on approach.
\vspace{0.2em}

\section*{Areas of Expertise}
\begin{center}
\renewcommand{\arraystretch}{1.2} % lebih padat
\begin{tabular}{p{0.3\textwidth} p{0.3\textwidth} p{0.3\textwidth}}
  \rowcolor[HTML]{F2F2F2} 
  \ding{51} Seismic Interpretation & \ding{51} Geophysical Survey Design & \ding{51} Project Planning  \\
  \rowcolor[HTML]{FFFFFF} 
  \ding{51} Subsurface Modeling & \ding{51} Data Acquisition & \ding{51} Signal Processing \\
  \rowcolor[HTML]{F2F2F2} 
  \ding{51} Python for Geoscience & \ding{51} Data Visualization & \ding{51} Quantitative Interpretation \\
  \rowcolor[HTML]{FFFFFF} 
  \ding{51} Scientific Writing & \ding{51} Geostatistics & \ding{51} Qualitative Interpretation \\
  \end{tabular}    
\end{center}

\section*{Education}

\noindent
\setlength{\fboxsep}{0pt}%
\colorbox[HTML]{F2F2F2}{%
  \parbox{\linewidth}{%
    \textbf{Master of Science, Physics}, Universitas Padjadjaran, Sumedang, West Java \hfill Expected 2025%
  }%
}
\begin{itemize}[left=1.5em, noitemsep, topsep=0pt]
    \item Focus: Computational Physics, Seismic Analysis, and Subsurface Modeling.
    \item Graduate research: Developing seismic-based disaster detection systems aligned with regional priorities.
\end{itemize}

\noindent
\setlength{\fboxsep}{0pt}%
\colorbox[HTML]{F2F2F2}{%
  \parbox{\linewidth}{%
    \textbf{Bachelor of Science, Geophysics}, Universitas Padjadjaran, Sumedang, West Java \hfill 2023%
  }%
}
\begin{itemize}[left=1.5em, noitemsep, topsep=0pt]
    \item Key topics: Seismic Interpretation, Earth Structure, Geophysical Instrumentation.
    \item Final project: Prototype development of real-time disaster early warning system using geophysical signals.
\end{itemize}


\vspace{0.3em}

\section*{Leadership Experience}
\noindent 
\setlength{\fboxsep}{0pt}%
\colorbox[HTML]{F2F2F2}{%
  \parbox{\linewidth}{%
    \textbf{Program Coordinator, Pembinaan Desa}, Himpunan Mahasiswa Geofisika Indonesia \hfill 2022%
  }%
}
\vspace{0.3em}
coordinated long-term rural outreach, trained disaster awareness to 100+ residents.
\begin{itemize}[left=1.5em, noitemsep, topsep=0pt]
    \item Led semester-long rural development program involving weekly activities.
    \item Conducted social mapping across 50+ households to design relevant engagement.
    \item Collaborated with academics and community leaders to implement public education.
    \item Delivered earthquake preparedness training to over 100 residents.
\end{itemize}

\noindent 
\setlength{\fboxsep}{0pt}%
\colorbox[HTML]{F2F2F2}{%
  \parbox{\linewidth}{%
    \textbf{Chair, Independence Day Celebration Committee}, Community Resident Group \hfill 2023%
  }%
}
\vspace{0.3em}
organized community-wide event in less than a week, fostering strong local engagement.
\begin{itemize}[left=1.5em, noitemsep, topsep=0pt]
    \item Initiated and led a team of 15 to organize a major Independence Day celebration.
    \item Managed logistics for parades, children's activities, and technical event preparation.
    \item Engaged over 50 households, strengthening neighborhood ties and inclusivity.
    \item Built strong emotional bonding among volunteers through a family-oriented approach.
\end{itemize}

\vspace{0.3em}

\section*{Professional Experience}
\noindent
\setlength{\fboxsep}{0pt}%
\colorbox[HTML]{F2F2F2}{%
  \parbox{\linewidth}{%
    \textbf{Geophysical Engineer},  PT. Surya Brinka Persada, West Java \hfill 2022%
  }%
}
\vspace{0.3em}
lead team for acquisition resistivity survey, succeed completing one week ahead of schedule.
\begin{itemize}[left=1.5em, noitemsep, topsep=0pt]
    \item Led a team of 5 in 2D resistivity survey acquisition.
    \item Successfully completed the project one week ahead of schedule.
    \item Received positive feedback from the client for the team's performance.  
\end{itemize}

\noindent
\setlength{\fboxsep}{0pt}%
\colorbox[HTML]{F2F2F2}{%
  \parbox{\linewidth}{%
    \textbf{Geoscience Surveyor }, Yayasan Sedekah Air, Jakarta \hfill 2022%
  }%
}
\vspace{0.3em}
give training t ocollague, completed survey in rural area in Central Java.
\begin{itemize}[left=1.5em, noitemsep, topsep=0pt]
    \item Trained 3 new surveyors in geophysical survey techniques.
    \item Successfully completed a survey in a rural area of Central Java.
    \item Received positive feedback from the team for effective training methods.
\end{itemize}

\vspace{0.3em}

\section*{Certificates \& Grant}
\noindent
TOEFL ITP Score: 527 \hfill Universitas Indonesia \\
Copyright: Educational Documentary Film \hfill Kemenkumham \\
Graduate Research Grant \hfill Kemendikbud \\
Graduate Scholarship Awardee \hfill JFLS, Disdik Jabar 

\vspace{0.3em}

\section*{Technical Skills}
\noindent
\textbf{Programming}: Python, C++, MATLAB \hspace{1em}
\textbf{Geoscience Software}: Res2Dinv, Petrel, Geopsy, QGIS, Surfer \hspace{1em}
\textbf{Documentation \& Reporting}: LaTeX, Scientific Writing \& Technical Reporting \hspace{1em}
\textbf{Project Management \& Collaboration}: Trello, Notion \hspace{1em}
\textbf{Operating System}: Linux (Ubuntu)

\end{document}