\documentclass[10pt,a4paper]{article}

% Margin lebih sempit agar space lebih banyak
\usepackage[left=1cm, right=1cm, top=1cm, bottom=1cm]{geometry}
\usepackage{titlesec}
\usepackage{enumitem}
\usepackage{hyperref}
\usepackage{tabularx}
\usepackage{xcolor}
\usepackage{colortbl}
\usepackage{array} % untuk tabular yang rapi
\usepackage{pifont} % di preamble
\usepackage{hyperref}
\usepackage[indonesian]{babel}
%---------------------------------------------------------------------------------

\usepackage{titlesec} % Allows creating custom \sections

% Format of the section titles
\titleformat{\section}
{\color[HTML]{1F4E79}\bfseries\scshape\large\raggedright} % Biru profesional
{}{0em}{}
\titlespacing{\section}{0pt}{10pt}{4pt}
%----------------------------------------------------

% Gunakan fontspec untuk font modern seperti Raleway
\usepackage[default]{raleway}

% Hilangkan spasi antar paragraf
\setlength{\parskip}{0pt}
\setlength{\parindent}{0pt}

\begin{document}

\begin{center}
  {\Huge{Rendy Ramadhana Wahyudi}}\\[0.5em]
    
  {\scshape \large Fisika Instrumentasi \quad | \quad Antusias Industri Energi }\\[0.5em]

  
  \rule{\textwidth}{0.4pt}\\[0.5em]
  \normalsize
  Bandung, Jawa Barat \quad | \quad 
  \href{mailto:rendy.rw.rw@gmail.com}{rendy.rw.rw@gmail.com} \quad | \quad
  \href{https://rendy-rw.github.io/}{rendy-rw.github.io} \quad | \quad
  \href{https://www.linkedin.com/in/rendywahyudi}{linkedin.com/in/rendywahyudi}
\end{center}

\section*{Profil Singkat}
Talenta muda yang percaya diri, adaptif, dan memiliki semangat belajar tinggi. Menyukai kerja tim, diskusi produktif, dan kontribusi aktif dalam proyek kolaboratif. Memiliki kemampuan komunikasi lisan dan tulisan yang baik, serta mampu mengintegrasikan keterampilan teknis dan pemikiran strategis untuk mencapai tujuan bersama secara efektif.

\section*{Kompetensi Utama}
\renewcommand{\arraystretch}{1.2} % lebih padat
\begin{tabularx}{\textwidth}{>{\raggedright\arraybackslash}X >{\raggedright\arraybackslash}X >{\raggedright\arraybackslash}X}
  \rowcolor[HTML]{EDF4FB} 
  \ding{51} Instrumentasi & \ding{51} Sensor & \ding{51} Sistem Kontrol \\
  \rowcolor[HTML]{FFFFFF} 
  \ding{51} Elektronika & \ding{51} Pengolahan Sinyal & \ding{51} Akuisisi Data \\
  \rowcolor[HTML]{EDF4FB} 
  \ding{51} Pemrograman & \ding{51} Komputasi Ilmiah & \ding{51} Desain Berbantuan Komputer \\
\end{tabularx}

\section*{Pengalaman Profesional}

\setlength{\fboxsep}{0pt}
\colorbox[HTML]{EDF4FB}{%
  \parbox{\linewidth}{%
    \textbf{Teknisi Geofisika (PKWT)}, PT. Surya Brinka Persada, Kab. Bandung \quad | \quad Sep 2024 – Sekarang
  }%
}

Memimpin survei geofisika berbasis resistivitas 2D di berbagai wilayah Jawa Barat.

\begin{itemize}[left=1.5em, noitemsep, topsep=0pt]
  \item Mengkoordinasikan tim lapangan (3 orang), mengatur peran teknis secara efisien untuk mempersingkat durasi survei.
  \item Menyelesaikan seluruh proyek sebelum tenggat dengan kualitas data tinggi dan laporan akhir yang diterima tanpa revisi.
\end{itemize}

\setlength{\fboxsep}{0pt}
\colorbox[HTML]{EDF4FB}{%
  \parbox{\linewidth}{%
    \textbf{Asisten Peneliti Instrumentasi (PKWT)}, Tim Riset Geofisika UNPAD, Sumedang \quad | \quad Jul – Agu 2023
  }%
}

Mengembangkan sistem dalam batas waktu ketat, menggunakan berbagai pendekatan untuk efektifikas operasional.

\begin{itemize}[left=1.5em, noitemsep, topsep=0pt]
  \item Melakukan simulasi performa antena guna efisiensi biaya dan material, memanfaatkan perangkat lunak sumber terbuka.
  \item Menyelesaikan purwarupa fungsional yang digunakan sebagai basis pengembangan riset lanjutan.
\end{itemize}

\section*{Pengalaman Kepemimpinan}
\noindent
\setlength{\fboxsep}{0pt}%
\colorbox[HTML]{EDF4FB}{%
  \parbox{\linewidth}{%
    \textbf{Koordinator Program Pemberdayaan Desa}, HIMA Geofisika PEDRA UNPAD \quad | \quad Sep 2021 - Feb 2022%
  }%
}

Mengkoordinasikan 10 anggota tim pada program jangka panjang kepada lebih dari 200 warga.

\begin{itemize}[left=1.5em, noitemsep, topsep=0pt]
  \item Menyelenggarakan pelatihan kesiapsiagaan gempa bumi berbasis pendekatan CBDRR, simulasi tanggap darurat \textit{mock drill} untuk efektivitas penyaluran informasi.
  \item Berhasil menumbuhkan kesadaran warga terhadap risiko bencana sebesar 80\% berdasarkan survei pra dan pasca kegiatan.
\end{itemize}

\setlength{\fboxsep}{0pt}
\colorbox[HTML]{EDF4FB}{%
  \parbox{\linewidth}{%
    \textbf{Wakil Ketua}, POA School of Leader XIII UNPAD \quad | \quad Oktober 2020 - November 2020%
  }%
}

Memimpin proyek aksi sosial untuk meningkatkan kesadaran masyarakat mengenai perubahan iklim.
\begin{itemize}[left=1.5em, noitemsep, topsep=0pt]
  \item Bertanggung jawab atas monitoring terhadap 7 divisi, menggunakan WBS dan Trello untuk memantau progres kegiatan.
  \item Berhasil menumbuhkan kesadaran terhadap isu lingkungan pada 116 peserta, dengan pencapaian keberhasilan sebesar 60–100\% berdasarkan 10 indikator utama proyek.
\end{itemize}

\section*{Pendidikan}

\setlength{\fboxsep}{0pt}%
\colorbox[HTML]{EDF4FB}{%
  \parbox{\linewidth}{%
    \textbf{Magister Fisika}, Universitas Padjadjaran, Jawa Barat \quad | \quad Nov 2023 – Agt 2025 (Perkiraan)%
  }%
}
\begin{itemize}[left=1.5em, noitemsep, topsep=0pt]
  \item \textbf{Proyek Tugas Akhir}: \textit{Pengembangan sistem deteksi bencana yang selaras dengan prioritas regional.}
\end{itemize}

\setlength{\fboxsep}{0pt}%
\colorbox[HTML]{EDF4FB}{%
  \parbox{\linewidth}{%
    \textbf{Sarjana Geofisika}, Universitas Padjadjaran, Jawa Barat \quad | \quad Agt 2019 – Jul 2023%
  }%
}
{\raggedright\sloppy
\begin{itemize}[left=1.5em, noitemsep, topsep=0pt]
  \item \textbf{Proyek Tugas Akhir}: \textit{Pengembangan prototipe jaringan sensor berbasis IoT}, diterbitkan pada \linebreak DOI: \href{https://doi.org/10.2991/978-2-38476-283-5_7}{10.2991/978-2-38476-283-5\_7}.
\end{itemize}
}

\section*{Sertifikat dan Hibah}

\textbf{Skor TOEFL ITP: 527}, Universitas Indonesia \quad | \quad Juli 2024 \\
\textbf{Hak Cipta: Film Dokumenter Edukasi}, Kementerian Hukum dan HAM RI \quad | \quad April 2023 \\
\textbf{Hibah Penelitian Pascasarjana}, Kementerian Pendidikan, Kebudayaan, Riset, dan Teknologi \quad | \quad Agustus 2024 \\
\textbf{Penerima Beasiswa Pascasarjana}, JFLS, Dinas Pendidikan Provinsi Jawa Barat \quad | \quad Februari 2025

\section*{Keahlian Teknis}

\textbf{Pemrograman}: Python, C++, MATLAB, pengolahan sinyal, dan otomasi sistem \quad | \quad
\textbf{Instrumentasi dan Sistem \linebreak Embedded}: STM32, Arduino, ESP32 \quad | \quad
\textbf{Perangkat Lunak Teknik}: LabVIEW, Proteus, KiCad, STM32Cube IDE, Arduino IDE\quad | \quad
\textbf{Dokumentasi dan Pelaporan}: LaTeX, Ms.Words, GoogleDoc \quad | \quad
\textbf{Manajemen Proyek dan Kolaborasi}: Trello, Notion, WBS (Work Breakdown Structure) \quad | \quad
\textbf{Sistem Operasi}: Windows, Linux (Ubuntu, Debian)

\end{document}
